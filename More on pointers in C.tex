\documentclass[twosides]{article}
\title{More on pointers in C}
\author{CSC315 Programming Language Concepts}
\date{04 October 2017}

\setlength{\parindent}{0pt}
\setlength{\parskip}{6pt}

\usepackage{listings}
\lstset{language=c}
\begin{document}
\maketitle

\section{Pointers to functions}

Move the code that constructs a calculator
from the \lstinline+main()+ function into
a separate function.

\begin{lstlisting}{}
#include <stdio.h>
#include <stdlib.h>

// Create a name for a pointer to a
// function that has two double precision
// floating point arguments and returns to
// its caller a double precision floating
// point value.
typedef double (*myFunctionPointer)(double a,double b) ;

// Here's a declaration of a pointer to
// a function that has two double arguments
// and returns a double to its caller.
double (*functionPointer)(double a, double b) ;

// Declare and define four simple functions.
double add( double x, double y ) {
  return x + y ;
} // add( double, double )

double subtract( double x, double y ) {
  return x - y ;
} // subtract( double, double )

double multiply( double x, double y ) {
  return x * y ;
} // multiply( double, double )

double divide( double x, double y ) {
  return x / y ;
} // divide( double, double )

// Define a data structure
// that contains pointers to functions.
struct calculator {
  myFunctionPointer ap ;
  myFunctionPointer sp ;
  myFunctionPointer mp ;
  myFunctionPointer dp ;
} ;


// Create aliases for the struct and for pointers
// to the struct.
typedef struct calculator Calculator, *CalculatorPointer ;


int main( int argc, char **argv ) {

  functionPointer = add ;

  printf( "size of function pointer = %d\n",
      sizeof(functionPointer)) ;
  printf( "3 + 4 = %8.4f\n", functionPointer(3,4)) ;

  // Create a struct that contains pointers to four functions.
  // Store the address of the struct in a pointer.
  CalculatorPointer cp =
      (CalculatorPointer) malloc(sizeof(Calculator)) ;

  // Initialize the struct by assigning
  // the addresses of four functions to
  // the function pointers in the struct.
  cp->ap = add ;
  cp->sp = subtract ;
  cp->mp = multiply ;
  cp->dp = divide ;


  printf( "\n\n" ) ;

  // Compute and print sum, difference, product,
  // and quotient. Show how to use a pointer
  // to a struct and pointers to functions.
  printf( "3 + 4 = %8.4f\n", cp->ap(3,4) ) ;
  printf( "3 - 4 = %8.4f\n", cp->sp(3,4) ) ;
  printf( "3 * 4 = %8.4f\n", cp->mp(3,4) ) ;
  printf( "3 / 4 = %8.4f\n", cp->dp(3,4) ) ;

  // Compute and print same results once
  // again, but this time showing another
  // way to use the pointers.
  printf( "\n\n" ) ;

  printf( "3 + 4 = %8.4f\n", (*cp).ap(3,4) ) ;
  printf( "3 - 4 = %8.4f\n", (*cp).sp(3,4) ) ;
  printf( "3 * 4 = %8.4f\n", (*cp).mp(3,4) ) ;
  printf( "3 / 4 = %8.4f\n", (*cp).dp(3,4) ) ;


} // main( int, char **  )
  \end{lstlisting}

\section{Binary search tree}

This program requires one or more integer
arguments on the command line.

Add a function that will compute the height
of the tree that the program constructs.
Use the \lstinline+printNodes()+ function as
a model.

\begin{lstlisting}{}
#include <stdio.h>
#include <stdlib.h>

struct node {
  int value ;
  struct node* lp;
  struct node* rp;
} ;

typedef struct node Node ;
typedef struct node* NodePointer;

NodePointer createRoot( int n ) {
  NodePointer np = (NodePointer) malloc( sizeof(Node) ) ;
  np->value = n ;
  np->lp = NULL ;
  np->rp = NULL ;
  return np ;
} // createRoot( int )

NodePointer addNode( NodePointer np, int n ) {
  NodePointer rp = np ;

  if( np == NULL ) {
    rp = createRoot( n ) ;
  } // if
  else if( n < np->value ) {
    if( np->lp == NULL ) {
      np->lp = createRoot( n ) ;
    } // if
    else {
      np->lp = addNode( np->lp, n ) ;
    } // else
  } // else if
  else if( n > np->value ) {
    if( np->rp == NULL ) {
      np->rp = createRoot( n ) ;
    } // if
    else {
      np->rp = addNode( np->rp, n ) ;
    } // else
  } // else if
  else {
    // if value to be added is already in tree, do nothing
    // (do not add duplicate values to tree)
  } // /else

  return rp;
} // addNode( NodePointer, int )

void printNodes( NodePointer np ) {
  if( np != NULL ) {
    printNodes( np->lp ) ;
    printf( "value at node = %4d\n", np->value ) ;
    printNodes( np->rp ) ;
  } // if
} // printNodes( NodePointer )

int main( int argc, char** argv ) {

  if( argc > 1 ) {

    NodePointer rp = createRoot( atoi(argv[1]) ) ;

    int i;
    for( i = 2; i < argc; i++ ) {
      rp = addNode( rp, atoi(argv[i]) ) ;
    } // for

    printNodes( rp ) ;
  } // if

} // main( int, char** )
  \end{lstlisting}

\end{document}
